\chapter*{Abstract}
\oddsidemargin = -30pt

This report presents the development of a domain-specific markup language for interactive storytelling.
The problem being dealt with is the need for a more intuitive and user-friendly DSL for creating interactive stories with branching paths and multiple outcomes. The language should incorporate declarative and imperative syntax, objects, variables, and collections, and require user input in the form of choices and interactions. User-generated content and various types of output can enhance the storytelling experience. The goal is to create a program that allows for dynamic, personalized stories that engage the reader and encourage active participation. 

One of the main benefits of domain-specific markup languages for interactive storytelling is that they free writers from having to be experts in programming or technical skills so they can concentrate on the content and structure of the story. Authors will be able to express story concepts like characters, settings, plot events, and narrative choices using a domain-specific language that is based on an easy-to-learn markup syntax.

An interactive storytelling programming language can be useful in different domains like: education, marketing, interactive fields, or virtual reality experiences. However, we decided to implement a DSL that will be able to generate pictures based on the input text. The users will write the story and with the help of AI, different images will be auto-generated. This can be used in video games with dynamic storylines that change based on the player's decisions and actions, while also providing visuals that match the story's mood and ambiance. Also, it can be used as a tool for creative writing, providing writers with visual inspiration that complements their text and helps to develop their storytelling skills. 

Generally, an interactive storytelling programming language with an AI that generates pictures about the text can provide a powerful tool for creating mesmerizing and engaging narratives across a variety of fields and applications, while also allowing for creative exploration and experimentation. It has the potential to revolutionize the way we create and consume stories, making them more engaging, immersive, and accessible to a wider audience.

\textbf{Keywords: }  DSL, storytelling, outcomes, branches, grammar, parse tree.