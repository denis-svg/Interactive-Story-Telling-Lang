\addcontentsline{toc}{chapter}{Conclusions}
\chapter*{Conclusions}

Domain Specific Language can be useful for different tasks, purposes and people. The difference between Interactive storytelling DSLs and other programming languages is that they must cope with outcomes, branches and characters. These notions are naturally understood by humans, but can be complicated to be transposed into code. This article introduces a new DSL concept, which develops the idea of a more friendly markup language for interactive storytelling.
Domain Specific Languages, or DSLs, are like a secret weapon in the world of programming. They're tailor-made for specific tasks, purposes, and people, acting as a useful tool for those specific situations. Now, where things get really interesting is when you start to look at Interactive storytelling DSLs.These are not like your average programming languages. They've got a tricky job on their hands. They need to deal with outcomes, branches, and characters - things that we, as humans, understand naturally, but are pretty tricky to translate into code. Imagine trying to explain to a robot how a choose-your-own-adventure book works - that's the challenge we're talking about here.That's why we're excited to introduce a fresh concept - a new DSL that's designed to make creating interactive stories a whole lot friendlier. We're talking about a language that speaks more like a human and less like a machine. This new DSL we're working on is like a markup language made especially for interactive storytelling.It's like we're trying to build a bridge between the world of human creativity and the somewhat cold, logical world of code. Our goal is to make creating complex, branching narratives as intuitive as telling a bedtime story. And while we're still in the early stages, we're super excited about the potential of this new DSL to revolutionize the world of interactive storytelling.
