\addcontentsline{toc}{chapter}{Introduction}
\chapter*{Introduction}

Interactive storytelling is a rapidly growing field that involves creating stories where the reader or player performs an active role in shaping the narrative. While there are numerous programming languages available for game and simulation development, creating interactive stories with different pathways and multiple outcomes can be a complex and time-consuming task. This is where a domain-specific language (DSL) can prove to be incredibly beneficial. A domain-specific language (DSL) is a programming language that is designed for a specific domain or task, rather than being a general-purpose language. A DSL that is specifically designed for interactive storytelling can provide domain experts, such as writers, game designers, and virtual reality enthusiasts, with a more intuitive and user-friendly way to create compelling stories with branching paths and dynamic outcomes. By using a DSL, domain experts can focus more on the creative aspects of storytelling and less on the technical details of programming, thereby enabling them to bring their ideas to life more efficiently and effectively. Furthermore, a DSL has more benefits, like: productivity, quality, validation and verification, data longevity, platform independence, and domain involvement[1].