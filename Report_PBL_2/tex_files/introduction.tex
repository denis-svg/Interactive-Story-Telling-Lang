\addcontentsline{toc}{chapter}{Introduction}
\chapter*{Introduction}
\oddsidemargin = -30pt
    Interactive storytelling is a rapidly growing field that involves creating stories where the reader or player performs an active role in shaping the narrative. While there are numerous programming languages available for game and simulation development, creating interactive stories with different pathways and multiple outcomes can be a complex and time-consuming task. This is where a domain-specific language (DSL) can prove to be incredibly beneficial. A domain-specific language (DSL) is a programming language that is designed for a specific domain or task, rather than being a general-purpose language. A DSL that is specifically designed for interactive storytelling can provide domain experts, such as writers, game designers, and virtual reality enthusiasts, with a more intuitive and user-friendly way to create compelling stories with branching paths and dynamic outcomes. By using a DSL, domain experts can focus more on the creative aspects of storytelling and less on the technical details of programming, thereby enabling them to bring their ideas to life more efficiently and effectively. Furthermore, a DSL has more benefits, like: productivity, quality, validation and verification, data longevity, platform independence, and domain involvement[1].
    
    Interactive storytelling is a fascinating space that's growing like wildfire. It's all about creating stories where you, the reader or player, get to have a say in how things turn out. But let's face it - weaving together complex narratives with multiple endings isn't a walk in the park, especially when you're dealing with generic programming languages.
    This is where a domain-specific language (DSL) comes into play. A DSL is like a special tool, built just for a specific job. In our case, a DSL designed for interactive storytelling could be a real lifesaver. It can make the task of crafting intricate stories easier and more intuitive, opening the door for writers, game designers, and even virtual reality fans to bring their stories to life.Think of it like this - with a DSL, you can focus more on creating amazing stories and less on wrestling with code. It's all about making the technical stuff less of a headache, so you can let your creative juices flow.Plus, there's more good news. A DSL can help you work faster and produce better quality stories. It ensures your narratives work as they should and makes it easier to keep your stories fresh and updated. It even allows your stories to run on different platforms, and invites more people to join in the fun of creating interactive narratives.
    
    So, if you're into interactive storytelling, a DSL might just be your new best friend. It's all about making your life easier, your stories better, and opening up the world of interactive storytelling to as many people as possible.
    
    
    
    
    \\
    \\

\addcontentsline{toc}{section}{Problem Statement}
\section*{Problem Statement}   

Markup language for interactive storytelling is a relatively new area of focus in the field of digital storytelling. The traditional narrative structure of linear storytelling is being challenged by the emergence of interactive storytelling, which allows the reader to participate in the story and influence its outcome. Markup languages are being developed to facilitate the creation and dissemination of interactive stories.
The problem with traditional linear storytelling is that it does not allow for much interaction from the reader. While the story may be entertaining, it can quickly become predictable and repetitive. Interactive storytelling, on the other hand, allows for the reader to be an active participant in the story, making choices and influencing the outcome.
 However, creating an interactive story can be a complex and time-consuming process. The story must be designed to allow for multiple paths and outcomes, and the technology used to create the story must be able to handle the complexity of the interactive elements. This is where markup languages come in.
Markup languages for interactive storytelling provide a way to structure the story in a way that allows for interactivity. These languages allow for the creation of branching paths, where the reader can make choices that affect the outcome of the story. They also allow for the creation of multimedia elements, such as sound, images, and video, to enhance the storytelling experience.
However, there are several challenges to developing markup languages for interactive storytelling. One challenge is ensuring that the language is easy to use and accessible to both experienced and novice storytellers. Another challenge is ensuring that the language is flexible enough to allow for a wide range of storytelling styles and genres.
In the end, the development of markup languages for interactive storytelling is an exciting area of innovation in the world of digital storytelling. As the technology continues to develop, we can expect to see more and more interactive stories being created and shared, providing readers with a more engaging and immersive storytelling experience.

\addcontentsline{toc}{section}{Solution Approach}
\section*{Solution Approach} 
\noindent What is the solution concept?

To address the need of creation of a markup language for story-telling the demand for which has surged in the past decade with popularization of visual novels, one can be developed from scratch, the goals of such language would be to provide authors with a toolset for creating engaging interactive stories that are easily shared and played by the audience.

The first step in developing is identifying the key elements that make up an interactive narrative. At high level such a story consists of a series of scenes or chapters that the player progresses through, with each scene offering a set of choices that can affect the story’s outcome. Therefore, the language must be capable of representing these elements, be easily understandable and machine readable.
One possible solution is to use XML-based format, with each scene represented as an individual element within the document. These scenes can be further broken down into sub-elements that represent the various narrative components, such as dialogue, actions and character descriptions. Similarly, the choices that the player makes can be represented as elements within each scene, with the outcome of each choice encoded as an attribute of that element.

To ensure that the language is both human-readable and machine-readable, it may be helpful to include annotations or comments within the markup. These annotations can help authors understand the structure and flow of the story, while also providing hints for how to use various elements and attributes. Additionally, including metadata such as the author's name and the date of creation can help to track the ownership and history of the story.

Once the basic structure of the markup language has been established, the next step is to define a set of rules for how the language should be used. This can include guidelines for naming conventions, data types, and formatting, as well as rules for how the language should be validated and parsed.

To ensure that the markup language is user-friendly, it may be useful to provide a set of tools or templates that authors can use to create and edit their stories. These tools could include a graphical user interface for creating and editing scenes, as well as a validator that checks the markup for errors and inconsistencies. Additionally, providing a set of pre-built components or templates that authors can use as a starting point for their stories can help to speed up the development process and ensure consistency across different stories.

One possible example of a successful markup language for interactive storytelling is Inkle's Ink software. Ink is a powerful, flexible language that allows authors to create complex, branching narratives with ease. It features a robust set of tools for editing and testing stories, as well as a built-in interpreter that can be used to play the stories on a variety of platforms. In creating our own markup language the strive would be to mimic with possible improvements the feature set of Inkle’s solution on interactive storytelling writing.

\noindent Who is the Target audience?

A target audience is a group of people that are most likely to be interested in our service or product. The target group for interactive storytelling can vary on the specific story, format, genre or way of implementation. However, interactive storytelling is often designed for people who are ready to engage completely in the action, and be allowed to participate in the storyline, rather than just passively consuming it.
Some examples of target groups for interactive storytelling might be:

 \begin{enumerate}
                 \item Children and young adults: Interactive storytelling can be a fun and educational way to engage younger audiences and get them excited about reading and storytelling. Many interactive storybooks and apps are designed with children in mind, using interactive elements like animation and sound effects to make the story come to life.
                 \item Adults: Even for mature people, this could be an interesting way of entertainment. As many people are into reading books, some of them would be more than happy to be able to influence the development of the story.
                 \item Gamers: Interactive storytelling is often a key component of video games, which can appeal to a wide range of ages and interests. Games that feature branching narratives, player choices, and other interactive elements are particularly popular among gamers who enjoy engaging storytelling experiences.
                 \item Virtual reality enthusiasts: Virtual reality (VR) technology allows users to step into and interact with immersive digital environments, making it a perfect medium for interactive storytelling. VR experiences can be particularly appealing to well-informed people in the use of modern technology, who are looking for new and innovative ways to engage with stories.
 \end{enumerate}

 To sum up, the target group for interactive storytelling can be quite broad, and may appeal to anyone who is looking for a more immersive and interactive way to experience stories where it will be possible to influence the development of the storyline.

\noindent What is our Motivation?
 
As a group of students who are passionate about computer science and storytelling, we have identified a gap in the current programming languages available that cater specifically to interactive storytelling. While there are many programming languages out there that allow users to create games or simulations, these languages don't always make it easy to create interactive stories that have branching paths and multiple endings.                                                    
We believe that interactive storytelling is a powerful tool for teaching and entertainment, and we want to make it more accessible to people who are interested in this field. We are motivated to develop a programming language that makes it easy for people to create interactive stories, regardless of their level of programming expertise.
We want to create a language that is intuitive, user-friendly, and flexible. It should allow users to easily create branching story paths, define character actions and dialogue, and create different outcomes based on user input. This language should also be easy to learn, so that even people who have no experience with programming can use it to create compelling interactive stories.
We are motivated to create this language because we believe that interactive storytelling has the potential to be a powerful educational tool. It can help students learn important concepts by engaging them in a narrative that allows them to explore different scenarios and outcomes. By making interactive storytelling more accessible, we hope to inspire more people to explore this exciting field and create engaging, interactive stories that can be shared with others.

\addcontentsline{toc}{section}{Benefits and Impact}
\section*{Benefits and Impact} 

\noindent What is the impact of the problem over the domain of study?

 Interactive storytelling programming languages, or ISPLs, are programming languages designed specifically for the creation of interactive narratives and games. These languages offer a unique and powerful tool for creating immersive and engaging experiences for users. The impact of ISPLs on the programming domain can be significant in several ways:

 \begin{enumerate}
                 \item Increased creativity: ISPLs offer a new way for programmers to express their creativity. With ISPLs, programmers can use their skills to create engaging narratives, characters, and interactive elements that keep users engaged and interested in the story. This can be a refreshing break from traditional programming, which often involves working on utilitarian projects.
                 \item Expansion of the programming domain: ISPLs can attract a new group of programmers who are interested in creating interactive stories and games. This can expand the programming domain and attract new talent to the industry. Additionally, ISPLs can allow developers to create new types of programs and applications that were not possible before, such as interactive educational tools, immersive simulations, and more.
                 \item Enhanced user engagement: ISPLs can help developers create interactive stories and games that are more engaging and immersive for users. By combining storytelling with programming, developers can create experiences that are more engaging than traditional software. This can be especially useful for educational tools, which can use interactive narratives to teach complex concepts in a more engaging way.
                 \item Improved user experience: ISPLs can help developers create applications with a more natural user interface. With interactive storytelling, users can navigate through the application in a more intuitive way, using conversation or exploration, which can lead to a better overall user experience.
 \end{enumerate}
 Overall, ISPLs have the potential to be a powerful tool for programmers, allowing them to create engaging interactive narratives and games that can expand the programming domain and offer users more engaging and immersive experiences.

\noindent What are the Competition and Alternatives?

 \begin{itemize}
                \item \textbf{Inky}
 \end{itemize}
 
Inky is a free, open-source tool designed to create interactive narrative scripts for video games. It is primarily used in the development of games that use the ink narrative scripting language, which was created by the game development studio, Inkle.
Inky is designed to make it easier for writers and developers to create and iterate on interactive narrative content without the need for programming expertise. It provides a user-friendly interface where writers can easily create, edit, and preview ink scripts. Inky also provides tools for managing branching paths, variables, and game logic.
In addition to its editing capabilities, Inky also provides features for version control, collaboration, and integration with game engines. It can output ink scripts in a variety of formats, including JSON, XML, and custom C# classes.
Overall, Inky is a powerful and flexible tool for creating interactive narratives for games, enabling writers and developers to work more efficiently and collaboratively in the creation of immersive game experiences.
\begin{itemize}
                \item \textbf{Yarn Spinner}
 \end{itemize}
 
 Yarn Spinner is an open-source tool designed to help game developers and writers create branching, non-linear narratives for video games. It was created by the game development studio, Secret Lab, and is free to use under the MIT license.
Yarn Spinner uses a scripting language called Yarn, which is designed to be easy to read and write, and allows writers to create complex branching paths and dialogue trees for their game's characters. The Yarn language is similar in structure to a simplified version of JavaScript, making it easy for developers with some programming experience to use.
In addition to its scripting capabilities, Yarn Spinner also includes a powerful dialogue engine that can handle complex conversations, conditional branching, and other advanced features. It can be integrated with various game engines, including Unity, Unreal Engine, and Godot.
Yarn Spinner also includes a number of features to help developers and writers collaborate on game narrative. It allows multiple writers to work on the same project simultaneously, provides version control, and includes a number of debugging and testing tools.
Overall, Yarn Spinner is a powerful and flexible tool for creating interactive narratives for video games, making it easier for developers and writers to create immersive, non-linear stories that engage players and keep them coming back for more.

 \begin{itemize}
                \item \textbf{StoryMapJS}
 \end{itemize}

 StoryMapJS is a free, open-source tool that allows users to create interactive maps with multimedia content, such as images, videos, and audio. It was created by the Knight Lab at Northwestern University and is designed to help journalists, educators, and others tell stories in an engaging and visually compelling way.
With StoryMapJS, users can create a sequence of locations on a map, and add multimedia content to each location. This can include images, videos, audio, and text. Users can also add annotations, captions, and other information to provide context and guide readers through the narrative.
StoryMapJS provides a simple, user-friendly interface for creating these interactive maps, and can be used by anyone with basic web development skills. The resulting map can be embedded on a website or shared as a standalone link, making it easy to distribute and share with others.
Overall, StoryMapJS is a powerful and flexible tool for creating interactive maps with multimedia content. It is ideal for educators, journalists, and others who want to tell stories in an engaging and visually compelling way, and can be used for a wide range of applications, from historical tours to travel guides to news stories.